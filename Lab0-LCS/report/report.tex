\documentclass[11pt]{article}

\usepackage{ctex}
\usepackage{bookmark}
\usepackage{listings}
\usepackage{epsfig}
\usepackage{lscape}
\usepackage{multirow}
\usepackage{longtable}
\usepackage{amsmath,amssymb,amsthm}
\usepackage{graphicx}
\usepackage{color}
\usepackage{placeins}
\usepackage{url}
\usepackage{cases}
\usepackage{hyperref}
\usepackage{setspace}
\usepackage{subfigure}
\usepackage{float}
\usepackage{listings}
\usepackage{framed}
\usepackage[c2]{optidef}
\usepackage{tikz}
\usepackage{caption}
\usepackage{algorithm}
\usepackage{algorithmic}
\usepackage{endnotes}
\usepackage{fontspec}


\usepackage{booktabs}

\oddsidemargin 0pt
\evensidemargin 0pt
\marginparwidth 10pt
\marginparsep 10pt
\topmargin -20pt
\textwidth 6.5in
\textheight 8.5in
\parindent = 20pt

\DeclareMathOperator*{\argmin}{argmin}
\DeclareMathOperator*{\minimax}{minimax}

\renewcommand{\algorithmicrequire}{ \textbf{function:}}
\renewcommand{\algorithmicreturn}{ \textbf{end function}}
\newcommand{\blue}[1]{\begin{color}{blue}#1\end{color}}
\newcommand{\magenta}[1]{\begin{color}{magenta}#1\end{color}}
\newcommand{\red}[1]{\begin{color}{red}#1\end{color}}
\newcommand{\green}[1]{\begin{color}{green}#1\end{color}}

\newtheorem{theorem}{Theorem}
\newtheorem{proposition}{Proposition}
\newtheorem{lemma}{Lemma}
\newtheorem{corollary}{Corollary}
\newtheorem{remark}{Remark}
\newtheorem{assumption}{Assumption}
\newtheorem{definition}{Definition}
%\newenvironment{proof}{{\noindent\it Proof}\quad}{\hfill $\square$\par}

%\usepackage{sidecap}


\definecolor{dkgreen}{rgb}{0,0.6,0}
\definecolor{gray}{rgb}{0.5,0.5,0.5}
\definecolor{mauve}{rgb}{0.58,0,0.82}
\setmonofont{Consolas}

\lstset{ %
	language=C++,           		% the language of the code
	basicstyle=\small\ttfamily,     % the size of the fonts that are used for the code
	numbers=left,                   % where to put the line-numbers
	numberstyle=\tiny\color{gray},  % the style that is used for the line-numbers
	numbersep=5pt,                  % how far the line-numbers are from the code
	backgroundcolor=\color{white},  % choose the background color. You must add \usepackage{color}
	showspaces=false,               % show spaces adding particular underscores
	showstringspaces=false,         % underline spaces within strings
	showtabs=false,                 % show tabs within strings adding particular underscores
	frame=single,                   % adds a frame around the code
	rulecolor=\color{black},        % if not set, the frame-color may be changed on line-breaks within not-black text (e.g. commens (green here))
	tabsize=2,                      % sets default tabsize to 2 spaces
	captionpos=b,                   % sets the caption-position to bottom
	breaklines=true,                % sets automatic line breaking
	breakatwhitespace=false,        % sets if automatic breaks should only happen at whitespace
	title=\lstname,                 % show the filename of files included with \lstinputlisting;
									% also try caption instead of title
	keywordstyle=\color{blue},      % keyword style
	commentstyle=\color{dkgreen},   % comment style
	stringstyle=\color{mauve},      % string literal style
	escapeinside={\%*}{*)},         % if you want to add LaTeX within your code
	morekeywords={*,...}            % if you want to add more keywords to the set
}


\begin{document}

\title{\bf Lab 0 实验报告}

\author{}
\date{\today}
\maketitle


\section{问题描述}

求解最长公共子序列,按字母序输出所有结果。


\section{解题思路}

用动态规划求解最长公共子序列长度,再根据状态转移方程反推输出子序列,输出要求字母序且不重复,使用$ set $结构存储结果。

\section{状态定义}

$ dpTable[i][j] $表示$ x[0:i] $与$ y[0:j] $的最长公共子序列长度,
$ x[0] $和$ y[0] $定义为空字符,意在$ dpTable $初始化时为0。


\section{状态转移方程}

\begin{itemize}
	\item 若$ x[i-1]=y[j-1] $,表示$ x $和$ y $最后一位相等,则$ dpTable[i][j]=dpTable[i-1][j-1]+1 $;
	\item 若$ x[i-1]\neq y[j-1] $,则$ dpTable[i][j]$为$ dpTable[i-1][j] $与$ dpTable[i][j-1] $中的最大值。
\end{itemize}

于是可得状态转移方程如下:

$$
	dpTable[i][j]=
	\begin{cases}
		0 & i==0\ or\ j==0 \\
		max(dpTable[i-1][j],dpTable[i][j-1]) & x[i-1]\neq y[j-1] \\
		dpTable[i-1][j-1]+1 & x[i-1]==y[j-1]
	\end{cases}
$$

由此编写动态规划部分代码如下:

\begin{lstlisting}
	int lcs(string a, string b, int m, int n) {
		dpTable = vector<vector<int>>(m + 1, vector<int>(n + 1));
		for (int i = 0; i < m + 1; ++i) {
			for (int j = 0; j < n + 1; ++j) {
				if (i == 0 || j == 0)
					dpTable[i][j] = 0;
				else if (a[i - 1] == b[j - 1])
					dpTable[i][j] = dpTable[i - 1][j - 1] + 1;
				else
					dpTable[i][j] = max(dpTable[i - 1][j], dpTable[i][j - 1]);
			}
		}
		return 0;
	}
\end{lstlisting}

时间复杂度为$ O(mn) $,$ for $循环执行$ (m+1)\times (n+1) $颗粒时间;

空间复杂度为$ O(mn) $,生成表格大小为$ (m+1)\times (n+1) $;

通讯复杂度为$ O\left(\frac{mn}{B}\right) $,内层循环$ cache miss $为$ \left(\frac{n}{B}+1+\frac{n}{B}\right) $;

程序优化方法:将传统递归解法优化为动态规划打表。


\section{结果输出}

根据状态转移方程反推可得子序列,结果输出部分代码如下:

\begin{lstlisting}
	int getlcs(int i, int j, string str) {
		while (i > 0 && j > 0) {
			if (x[i - 1] == y[j - 1]) {
				str.push_back(x[i - 1]);
				--i;
				--j;
			}
			else if (dpTable[i - 1][j] > dpTable[i][j - 1])
				--i;
			else if (dpTable[i - 1][j] < dpTable[i][j - 1])
				--j;
			else {
				getlcs(i - 1, j, str);
				getlcs(i, j - 1, str);
				return 0;
			}
		}
		reverse(str.begin(), str.end());
		lcsSet.insert(str);
		return 0;
	}
\end{lstlisting}

此处用尾递归解决路径分叉和降低时间复杂度,再用$ set $结构去重以及保持字母升序。

程序正确性:能力范围内的测试均通过,其他的靠自信。


\end{document}


